  const LevelingthePlayingField = `<div class="container" style="width: 100%; max-width: 800px;   box-shadow: 0 0 0.5rem 0.5rem rgba(255, 255, 255, 0.03); border-radius: 12px; box-shadow: 0 10px 15px rgba(0, 0, 0, 0.2); padding: 30px; box-sizing: border-box;">
        <h1 style="font-size: 2.25rem; font-weight: 700; color: #e6005c; margin-bottom: 20px; text-align: center; border-bottom: 2px solid #e6005c; padding-bottom: 15px;">Youth Leadership Initiative Guide</h1>

        <div style="margin-bottom: 30px;">
            <h2 style="font-size: 1.5rem; font-weight: 600; color: #e6005c; margin-top: 30px; margin-bottom: 15px; padding-left: 10px; border-left: 4px solid #e6005c;">Key Questions for Youth-Led Initiatives</h2>
            <ul style="list-style: none; padding: 0; margin-bottom: 20px;">
                <li style="  box-shadow: 0 0 0.5rem 0.5rem rgba(255, 255, 255, 0.03); padding: 15px 20px; margin-bottom: 10px; border-radius: 8px; box-shadow: 0 4px 6px rgba(0, 0, 0, 0.1); display: flex; align-items: flex-start;">
                    <span class="question-number" style="font-weight: 700; color: #e6005c; margin-right: 15px; font-size: 1.125rem; min-width: 30px; text-align: right;">01.</span>
                    <p style="margin: 0; color: #e2e8f0;"><strong>Who are the members of your tribe?</strong> List the people you already work with/anticipate partnering with who would likely welcome youth leadership (and would therefore bring the right energy and perspectives to youth-led initiatives).</p>
                </li>
                <li style="  box-shadow: 0 0 0.5rem 0.5rem rgba(255, 255, 255, 0.03); padding: 15px 20px; margin-bottom: 10px; border-radius: 8px; box-shadow: 0 4px 6px rgba(0, 0, 0, 0.1); display: flex; align-items: flex-start;">
                    <span class="question-number" style="font-weight: 700; color: #e6005c; margin-right: 15px; font-size: 1.125rem; min-width: 30px; text-align: right;">02.</span>
                    <p style="margin: 0; color: #e2e8f0;"><strong>Characteristics of collaborators:</strong> List some skills, attributes, mentorship/facilitator experience they possess that will best align your program’s goals.</p>
                </li>
                <li style="  box-shadow: 0 0 0.5rem 0.5rem rgba(255, 255, 255, 0.03); padding: 15px 20px; margin-bottom: 10px; border-radius: 8px; box-shadow: 0 4px 6px rgba(0, 0, 0, 0.1); display: flex; align-items: flex-start;">
                    <span class="question-number" style="font-weight: 700; color: #e6005c; margin-right: 15px; font-size: 1.125rem; min-width: 30px; text-align: right;">03.</span>
                    <p style="margin: 0; color: #e2e8f0;"><strong>Auditing and contextualization:</strong> What are the social, organizational, temporal, and geographical contexts that may influence how you work with youth researchers?</p>
                </li>
                <li style="  box-shadow: 0 0 0.5rem 0.5rem rgba(255, 255, 255, 0.03); padding: 15px 20px; margin-bottom: 10px; border-radius: 8px; box-shadow: 0 4px 6px rgba(0, 0, 0, 0.1); display: flex; align-items: flex-start;">
                    <span class="question-number" style="font-weight: 700; color: #e6005c; margin-right: 15px; font-size: 1.125rem; min-width: 30px; text-align: right;">04.</span>
                    <p style="margin: 0; color: #e2e8f0;"><strong>Communal spaces for collaboration:</strong> What spaces are currently available/needed in order to ideate, workshop, and develop healthy relationships with youth researchers?</p>
                </li>
                <li style="  box-shadow: 0 0 0.5rem 0.5rem rgba(255, 255, 255, 0.03); padding: 15px 20px; margin-bottom: 10px; border-radius: 8px; box-shadow: 0 4px 6px rgba(0, 0, 0, 0.1); display: flex; align-items: flex-start;">
                    <span class="question-number" style="font-weight: 700; color: #e6005c; margin-right: 15px; font-size: 1.125rem; min-width: 30px; text-align: right;">05.</span>
                    <p style="margin: 0; color: #e2e8f0;"><strong>Identifying needs of youth:</strong> What factors affect their abilities to show up and win? List ways you can support them to encourage strong teamwork and celebrate interdependence.</p>
                </li>
                <li style="  box-shadow: 0 0 0.5rem 0.5rem rgba(255, 255, 255, 0.03); padding: 15px 20px; margin-bottom: 10px; border-radius: 8px; box-shadow: 0 4px 6px rgba(0, 0, 0, 0.1); display: flex; align-items: flex-start;">
                    <span class="question-number" style="font-weight: 700; color: #e6005c; margin-right: 15px; font-size: 1.125rem; min-width: 30px; text-align: right;">06.</span>
                    <p style="margin: 0; color: #e2e8f0;"><strong>Sharing and storytelling:</strong> When the work is complete: How will you disseminate the learnings that emerge from the work? What audiences should you engage whose decision-making directly affects youth? What is the role of digital storytelling in elevating youths’ insights on issues and activities that impact them? Consider the purposes of dissemination (e.g., credentialing and cosigning youth, building community capacity for future youth-led work, shattering damaging stereotypes about marginalized identity groups).</p>
                </li>
            </ul>
        </div>

        <div class="note-section" style="  box-shadow: 0 0 0.5rem 0.5rem rgba(255, 255, 255, 0.03); border-radius: 8px; padding: 20px; margin-top: 30px; border-left: 5px solid #e6005c; color: #e2e8f0;">
            <p style="margin: 0; font-style: italic; font-size: 0.95rem;">We hope these examples spark conversations about the considerations and state of mind required to successfully engage young people as indispensable collaborators.</p>
            <p style="margin-top: 10px; margin-bottom: 15px; font-style: italic; font-size: 0.95rem;">Although there are many ways that teams engage youth in leadership initiatives, we found the following three practices to be effective in our work:</p>
            <ul style="list-style: disc; padding-left: 20px; margin-top: 15px; margin-bottom: 20px;">
                <li style="background-color: transparent; box-shadow: none; padding: 5px 0; margin-bottom: 5px; color: #a0aec0; display: list-item;">
                    <p style="margin: 0;"><strong>Socialization Practices:</strong> How to use an icebreaker with your group to get to know one another. How to create an environment that fosters trust and psychological safety. How to facilitate difficult conversations.</p>
                </li>
                <li style="background-color: transparent; box-shadow: none; padding: 5px 0; margin-bottom: 5px; color: #a0aec0; display: list-item;">
                    <p style="margin: 0;"><strong>Vulnerability:</strong> How to show vulnerability without oversharing. How to model vulnerability.</p>
                    <ul style="list-style: circle; padding-left: 20px; margin-top: 5px; margin-bottom: 0;">
                        <li style="background-color: transparent; box-shadow: none; padding: 2px 0; margin-bottom: 2px; color: #a0aec0; display: list-item;">
                            <p style="margin: 0;">Lead the way: How to make it a learning moment?</p>
                        </li>
                        <li style="background-color: transparent; box-shadow: none; padding: 2px 0; margin-bottom: 2px; color: #a0aec0; display: list-item;">
                            <p style="margin: 0;">Don't just hit the goal! How to make it a learning moment?</p>
                        </li>
                    </ul>
                </li>
                <li style="background-color: transparent; box-shadow: none; padding: 5px 0; margin-bottom: 5px; color: #a0aec0; display: list-item;">
                    <p style="margin: 0;"><strong>Stay away from buzzwords</strong></p>
                </li>
            </ul>
        </div>

        <div class="additional-considerations" style="  box-shadow: 0 0 0.5rem 0.5rem rgba(255, 255, 255, 0.03); border-radius: 12px; padding: 25px; margin-top: 40px; box-shadow: 0 5px 10px rgba(0, 0, 0, 0.15);">
            <h2 style="font-size: 1.5rem; font-weight: 600; border-left-color: #fc8181; color: #fed7d7; margin-top: 0; margin-bottom: 15px; padding-left: 10px; border-left: 4px solid #fc8181;">Additional Considerations</h2>
            <ul style="list-style: none; padding: 0; margin-top: 20px; margin-bottom: 20px;">
                <li style="  box-shadow: 0 0 0.5rem 0.5rem rgba(255, 255, 255, 0.03); color: #e2e8f0; padding: 12px 18px; margin-bottom: 8px; border-radius: 6px; box-shadow: none; display: flex; align-items: flex-start;">
                    <span class="question-number" style="font-weight: 700; color: #e6005c; margin-right: 15px; font-size: 1.125rem; min-width: 30px; text-align: right;">01.</span>
                    <p style="margin: 0;"><strong>Use language that better connects with the youth.</strong> If specific terminology is necessary:</p>
                    <ul style="list-style: disc; padding-left: 20px; margin-top: 5px; margin-bottom: 0;">
                        <li style="background-color: transparent; box-shadow: none; padding: 5px 0; margin-bottom: 5px; color: #e2e8f0; display: list-item;">
                            <p style="margin: 0;">Provide glossaries or include definitions within the content</p>
                        </li>
                        <li style="background-color: transparent; box-shadow: none; padding: 5px 0; margin-bottom: 5px; color: #e2e8f0; display: list-item;">
                            <p style="margin: 0;">Co-develop definitions of terms with the youth to create a shared language and understanding</p>
                        </li>
                    </ul>
                </li>
                <li style="  box-shadow: 0 0 0.5rem 0.5rem rgba(255, 255, 255, 0.03); color: #e2e8f0; padding: 12px 18px; margin-bottom: 8px; border-radius: 6px; box-shadow: none; display: flex; align-items: flex-start;">
                    <span class="question-number" style="font-weight: 700; color: #e6005c; margin-right: 15px; font-size: 1.125rem; min-width: 30px; text-align: right;">02.</span>
                    <p style="margin: 0;"><strong>Evaluate the effects of the provided communal space.</strong></p>
                    <ul style="list-style: disc; padding-left: 20px; margin-top: 5px; margin-bottom: 0;">
                        <li style="background-color: transparent; box-shadow: none; padding: 5px 0; margin-bottom: 5px; color: #e2e8f0; display: list-item;">
                            <p style="margin: 0;">How does the environment affect the power dynamics between adults and young people?</p>
                        </li>
                        <li style="background-color: transparent; box-shadow: none; padding: 5px 0; margin-bottom: 5px; color: #e2e8f0; display: list-item;">
                            <p style="margin: 0;">Do youth feel safe to express their honest opinions in the space?</p>
                        </li>
                        <li style="background-color: transparent; box-shadow: none; padding: 5px 0; margin-bottom: 5px; color: #e2e8f0; display: list-item;">
                            <p style="margin: 0;">If not, how can we pivot?</p>
                        </li>
                    </ul>
                </li>
                <li style="  box-shadow: 0 0 0.5rem 0.5rem rgba(255, 255, 255, 0.03); color: #e2e8f0; padding: 12px 18px; margin-bottom: 8px; border-radius: 6px; box-shadow: none; display: flex; align-items: flex-start;">
                    <span class="question-number" style="font-weight: 700; color: #e6005c; margin-right: 15px; font-size: 1.125rem; min-width: 30px; text-align: right;">03.</span>
                    <p style="margin: 0;"><strong>Give youth-led initiatives time to take shape.</strong> As Bruce Lee suggested, “Be water.” Working with youth who are learning about the research process requires fluidity to ensure the project can continue despite the challenges that come with this work (e.g., accepting that there can be a long work period before seeing the fruits of their labor; learning to situate their research projects within existing literature; making critical decisions around research design, data collection, and interpretation). Expect to take youth through several rounds of feedback before their insights can be fully realized.</p>
                </li>
            </ul>
        </div>
    </div>`;

    *******************************************
    **************************************
     const EnergizingYouthtoLead = ` <div style="  display: flex; flex-direction: column; align-items: center; justify-content: center; padding: 3rem 1rem;">
        <div style=" width: 100%;   box-shadow: 0 4px 10px rgba(0, 0, 0, 0.3);  border-radius: 0.75rem;   border: 1px solid rgba(0, 0, 0, 0.4); padding: 2.5rem;">
            <!-- Header Section -->
            <header style="text-align: center; margin-bottom: 2.5rem;">
                <h1 style="font-size: 3rem; line-height: 1.25; font-weight: 800; color: #ff0066; margin-bottom: 1rem;">
                    Energizing Youth to Lead
                </h1>
                <p style="font-size: 1.25rem; line-height: 1.75rem; color: #cccccc;">
                    Culturally Informed Motivation Strategies for Youth Leadership
                </p>
                <p style="margin-top: 1rem; font-size: 1rem; line-height: 1.5rem; color: #cccccc; max-width: 42rem; margin-left: auto; margin-right: auto;">
                    To effectively motivate young people, it's essential to first understand what drives them and then tailor opportunities accordingly. Here’s what our youth have identified as key energizers:
                </p>
            </header>

            <!-- Strategies Grid -->
            <div style="display: grid; grid-template-columns: 1fr; gap: 2rem;">
                <!-- On medium screens and up, use 2 columns -->
                <style>
                    @media (min-width: 768px) {
                        div[style*="grid-template-columns: 1fr"] {
                            grid-template-columns: repeat(2, 1fr);
                        }
                    }
                </style>

                <!-- Strategy 1: Empowering Choices and Real-World Engagement -->
                <div style="  box-shadow: 0 4px 10px rgba(0, 0, 0, 0.3); border:2px solid black;  padding: 1.5rem; border-radius: 0.5rem;  transition-property: box-shadow; transition-duration: 300ms;">
                    <h2 style="font-size: 1.5rem; line-height: 2rem; font-weight: 700; color: #2563eb; margin-bottom: 0.75rem; display: flex; align-items: center;">
                        <svg style="width: 1.5rem; height: 1.5rem; margin-right: 0.5rem; color: #3b82f6;" fill="none" stroke="currentColor" viewBox="0 0 24 24" xmlns="http://www.w3.org/2000/svg"><path stroke-linecap="round" stroke-linejoin="round" stroke-width="2" d="M13 10V3L4 14h7v7l9-11h-7z"></path></svg>
                        Empowering Choices & Real-World Engagement
                    </h2>
                    <ul style="list-style-type: disc; list-style-position: inside; color: cccccc; margin-top: 0.5rem; line-height: 1.5rem;">
                        <li style="margin-bottom: 0.5rem;">
                            <strong style="font-weight: 600;">Offer a selection of roles and projects:</strong> When students have more choices to explore their interests and strengths, they are more likely to find something they are passionate about.
                        </li>
                        <li>
                            <strong style="font-weight: 600;">Partner with youth in real-world settings:</strong> Projects that allow them to move out of the traditional classroom setting and into more dynamic environments (like professional co-working spaces) have been particularly energizing.
                        </li>
                    </ul>
                </div>

                <!-- Strategy 2: Cultivating Connection and "Twin-ergy" -->
                <div style="box-shadow: 0 4px 10px rgba(0, 0, 0, 0.3); border:2px solid black; padding: 1.5rem; border-radius: 0.5rem; box-shadow: 0 4px 6px -1px rgba(0, 0, 0, 0.1), 0 2px 4px -1px rgba(0, 0, 0, 0.06); transition-property: box-shadow; transition-duration: 300ms;">
                    <h2 style="font-size: 1.5rem; line-height: 2rem; font-weight: 700; color: #059669; margin-bottom: 0.75rem; display: flex; align-items: center;">
                        <svg style="width: 1.5rem; height: 1.5rem; margin-right: 0.5rem; color: #10b981;" fill="none" stroke="currentColor" viewBox="0 0 24 24" xmlns="http://www.w3.org/2000/svg"><path stroke-linecap="round" stroke-linejoin="round" stroke-width="2" d="M17 20h5v-2a3 3 0 00-5.356-1.857M17 20H2v-2a3 3 0 015.356-1.857M17 20v-2c0-.653-.165-1.272-.475-1.828M14 10a6 6 0 00-6-6H3a6 6 0 00-6 6v2a6 6 0 006 6h5a6 6 0 006-6v-2z"></path></svg>
                        Cultivating Connection & "Twin-ergy"
                    </h2>
                    <ul style="list-style-type: disc; list-style-position: inside; color: #cccccc; margin-top: 0.5rem; line-height: 1.5rem;">
                        <li style="margin-bottom: 0.5rem;">
                            <strong style="font-weight: 600;">Pair youth with peers they already enjoy spending time with:</strong> Youth are more likely to be enthusiastic about participating in activities that involve their friends, boosting morale and engagement. This is what our youth call <span style="font-weight: 700; color: #047857;">twin-ergy</span>.
                        </li>
                        <li>
                            <strong style="font-weight: 600;">Carve out time for friendships to emerge:</strong> Friends provide emotional support, alleviate anxiety with adults, and build confidence, helping youth step outside their comfort zones.
                        </li>
                    </ul>
                </div>

                <!-- Strategy 3: Infusing Fun and Flexibility -->
                <div style="box-shadow: 0 4px 10px rgba(0, 0, 0, 0.3); border:2px solid black; padding: 1.5rem; border-radius: 0.5rem; box-shadow: 0 4px 6px -1px rgba(0, 0, 0, 0.1), 0 2px 4px -1px rgba(0, 0, 0, 0.06); transition-property: box-shadow; transition-duration: 300ms;">
                    <h2 style="font-size: 1.5rem; line-height: 2rem; font-weight: 700; color: #ea580c; margin-bottom: 0.75rem; display: flex; align-items: center;">
                        <svg style="width: 1.5rem; height: 1.5rem; margin-right: 0.5rem; color: #f97316;" fill="none" stroke="currentColor" viewBox="0 0 24 24" xmlns="http://www.w3.org/2000/svg"><path stroke-linecap="round" stroke-linejoin="round" stroke-width="2" d="M14.828 14.828a4 4 0 01-5.656 0M9 10h.01M15 10h.01M21 12a9 9 0 11-18 0 9 9 0 0118 0z"></path></svg>
                        Infusing Fun & Flexibility
                    </h2>
                    <ul style="list-style-type: disc; list-style-position: inside; color: #cccccc; margin-top: 0.5rem; line-height: 1.5rem;">
                        <li style="margin-bottom: 0.5rem;">
                            <strong style="font-weight: 600;">Make time for fun:</strong> Projects incorporating travel, creativity, and active participation bring more joy and engagement. Avoiding long periods of inactivity and integrating enjoyable activities helps maintain enthusiasm.
                        </li>
                        <li style="margin-bottom: 0.5rem;">
                            <strong style="font-weight: 600;">Build in downtime:</strong> Provide space for youth to engage with peers and have fun outside scheduled activities. This was a major strategy for improving engagement.
                        </li>
                        <li>
                            <strong style="font-weight: 600;">Following the Energy of Youth:</strong> Allowing students to choose food/snacks and trying new foods together provides relaxation, increasing engagement and enjoyment.
                        </li>
                    </ul>
                </div>

                <!-- Strategy 4: Iterative Design and Continuous Improvement -->
                <div style="box-shadow: 0 4px 10px rgba(0, 0, 0, 0.3); border:2px solid black;padding: 1.5rem; border-radius: 0.5rem; box-shadow: 0 4px 6px -1px rgba(0, 0, 0, 0.1), 0 2px 4px -1px rgba(0, 0, 0, 0.06); transition-property: box-shadow; transition-duration: 300ms;">
                    <h2 style="font-size: 1.5rem; line-height: 2rem; font-weight: 700; color: #9333ea; margin-bottom: 0.75rem; display: flex; align-items: center;">
                        <svg style="width: 1.5rem; height: 1.5rem; margin-right: 0.5rem; color: #a855f7;" fill="none" stroke="currentColor" viewBox="0 0 24 24" xmlns="http://www.w3.org/2000/svg"><path stroke-linecap="round" stroke-linejoin="round" stroke-width="2" d="M4 4v5h.582m15.356 2A8.001 8.001 0 004 16.087V18m-9.426-8.863A8.001 8.001 0 0118 7.913V6m-1.426 8.863A8.001 8.001 0 0020 12.087V10m-9.426 8.863A8.001 8.001 0 014 16.087V18"></path></svg>
                        Iterative Design & Continuous Improvement
                    </h2>
                    <ul style="list-style-type: disc; list-style-position: inside; color: #cccccc; margin-top: 0.5rem; line-height: 1.5rem;">
                        <li style="margin-bottom: 0.5rem;">
                            <strong style="font-weight: 600;">Keep it fresh:</strong> Engaging youth is an iterative process. Focus on actually participating students, not past visions. Projects should be dynamic and open to continual evolution to remain relevant.
                        </li>
                        <li style="margin-bottom: 0.5rem;">
                            <strong style="font-weight: 600;">Orient volunteers and other stakeholders:</strong> Explain the iterative nature of the project to avoid misunderstandings and preserve a dynamic learning environment.
                        </li>
                        <li>
                            <strong style="font-weight: 600;">Collect and utilize feedback:</strong> Find ways to reach both introverted and extroverted youth (confidential surveys, discussions) to gather opinions and suggestions freely.
                        </li>
                    </ul>
                </div>

            </div>

            <!-- Conclusion Section -->
            <footer style="margin-top: 2.5rem; text-align: center; color: #cccccc;">
                <p style="font-size: 1.125rem; line-height: 1.75rem; max-width: 42rem; margin-left: auto; margin-right: auto;">
                    By aligning our support with the interests and enthusiasm of students, we create an environment where our young people felt valued and affirmed. Beyond providing resources, this is about actively participating in and celebrating the students’ journey.
                </p>
                <p style="margin-top: 1rem; font-size: 1rem; line-height: 1.5rem; max-width: 42rem; margin-left: auto; margin-right: auto;">
                    By following youths’ energy, providing essential resources, and fostering a supportive and fun atmosphere, adults can help shape a generation of confident, capable leaders. It’s about recognizing the unique contributions of each student, and creating opportunities that resonate with their passions and interests.
                </p>
            </footer>
        </div>
    </div>`;

    **********************************************
    *********************************************


    const QualityEngagementOverQuantityEngagemen = ` <div style=" display: flex; flex-direction: column; align-items: center; justify-content: center; padding: 3rem 1rem;">
        <div style=" width: 100%;  border-radius: 0.75rem; box-shadow: inset 0 1px 5px rgba(189, 189, 189, 0.1); padding: 2.5rem;">
            <!-- Header Section -->
            <header style="text-align: center; margin-bottom: 2.5rem;">
                <h1 style="font-size: 3rem; line-height: 1.25; font-weight: 800; color: #ff0066; margin-bottom: 1rem;">
                    Youth Leadership Journeys
                </h1>
                <p style="font-size: 1.25rem; line-height: 1.75rem; color: #c9c9c9; max-width: 50rem; margin-left: auto; margin-right: auto;">
                    This section presents five leadership stories of youths’ journeys while working with us, along with how and why the quality over quantity philosophy we employed positively impacted their sense of connection to the initiatives they led.
                </p>
                <ul style="list-style-type: disc; list-style-position: inside; text-align: left; margin-top: 1.5rem; max-width: 45rem; margin-left: auto; margin-right: auto; color: #c9c9c9; line-height: 1.6;">
                    <li>finding ways to schedule around other valued commitments that occupy youths’ time,</li>
                    <li>identifying ways to continually challenge the same groups of youth with new leadership initiatives over time,</li>
                    <li>valuing the many forms that youths’ contributions may take, and</li>
                    <li>building in ample reflection and planning time so as to be responsive to their suggestions on how working sessions should be structured.</li>
                </ul>
            </header>

            <!-- Story 1: Transitioning to the Off-Season -->
            <div style="box-shadow: 0 4px 10px rgba(0, 0, 0, 0.3); border:1px solid gray; padding: 2rem; border-radius: 0.75rem; box-shadow: 0 4px 6px -1px rgba(0, 0, 0, 0.1), 0 2px 4px -1px rgba(0, 0, 0, 0.06); margin-bottom: 2.5rem; display: flex; flex-direction: column; align-items: center; text-align: center;">
                <h2 style="font-size: 2.25rem; line-height: 2.5rem; font-weight: 700; color: #ff0066; margin-bottom: 1.5rem;">
                    Story 1: Transitioning to the Off-Season
                </h2>
                <img src="https://res.cloudinary.com/dpcxwe6gm/image/upload/v1752677431/Screenshot_19_tusert.png" alt="Malachi's Story Image" style="max-width: 100%; height: auto; border-radius: 0.5rem; margin-bottom: 1.5rem; box-shadow: 0 4px 6px -1px rgba(0, 0, 0, 0.1), 0 2px 4px -1px rgba(0, 0, 0, 0.06);">
                <p style="font-size: 1rem; line-height: 1.6; color: #c9c9c9; text-align: left;">
                    We began working with Malachi when he was in the 6th grade after he made an educational video to teach his community about Black sports icons. Malachi is a football player, basketball player, and track and field athlete who is a deep thinker. His teachers describe him as a student who sticks with projects, even when concepts are difficult for him to understand. We invited him and a small group of his peers to spend the summer working with us—turning their educational videos into professional-quality public service announcements that could be displayed publicly and used as an instructional resource for teachers during Black History Month. Malachi worked meticulously to clarify the exact angle he wanted to shoot the video from, and we encouraged him to do so. However, family and athletic commitments prevented him from devoting himself exclusively to our summer institute, and as a result the project never saw the light of day.
                </p>
                <p style="font-size: 1rem; line-height: 1.6; color: #c9c9c9; text-align: left; margin-top: 1rem;">
                    Over the course of middle school Malachi slowly worked his way up within our program—first as a photographer, then at an information table during community events while other students’ completed projects were publicly recognized. By 8th grade, Malachi was a star football athlete who was invited to play in the All-American Middle School football championships. It was clear that we would not be able to engage Malachi deeply in youth-voice initiatives during the Fall semester.
                </p>
                <p style="font-size: 1rem; line-height: 1.6; color: #c9c9c9; text-align: left; margin-top: 1rem;">
                    Spring semester was a different story! Off-season for Malachi’s athletic activities meant in-season for Black and Belonging! So this honor roll student-athlete was all in—engaging in weekly brainstorming sessions for new youth-led initiatives, conducting evaluations of the impact of our previous youth-led initiatives on students’ belonging and engagement, and presenting at conferences across the country to present logic models that teach scholars how to design youth-led initiatives that were similar to the ones he engaged in. By honoring Malachi’s natural rhythm of engagement and availability, and by being creative about ways athletes could be engaged in youth-led initiatives, Malachi’s peers also benefitted—not only from his strong leadership qualities, but also from his example of how life lessons learned in sports can translate into other arenas off the field.
                </p>
                <p style="font-size: 1rem; line-height: 1.6; color: #c9c9c9; text-align: left; margin-top: 1rem;">
                    And Malachi’s rhythm of engagement and availability has had even more far-reaching impacts because it got us thinking about ways other athletes could be engaged in youth-led initiatives in their off season.
                </p>
            </div>

            <!-- Story 2: Investing in the Franchise Player -->
            <div style="box-shadow: 0 4px 10px rgba(0, 0, 0, 0.3); border:1px solid gray;  padding: 2rem; border-radius: 0.75rem; box-shadow: 0 4px 6px -1px rgba(0, 0, 0, 0.1), 0 2px 4px -1px rgba(0, 0, 0, 0.06); display: flex; flex-direction: column; align-items: center; text-align: center;">
                <h2 style="font-size: 2.25rem; line-height: 2.5rem; font-weight: 700; color: #ff0066; margin-bottom: 1.5rem;">
                    Story 2: Investing in the Franchise Player
                </h2>
                <img src="https://res.cloudinary.com/dpcxwe6gm/image/upload/v1752677610/Screenshot_20_dmvn5e.png" alt="Mikal's Story Image" style="max-width: 100%; height: auto; border-radius: 0.5rem; margin-bottom: 1.5rem; box-shadow: 0 4px 6px -1px rgba(0, 0, 0, 0.1), 0 2px 4px -1px rgba(0, 0, 0, 0.06);">
                <p style="font-size: 1rem; line-height: 1.6; color: #c9c9c9; text-align: left;">
                    While Mikal was in high school, he was undoubtedly the franchise player for our youth-led initiatives. His exceptional ability to articulate and clarify the larger vision and purpose behind his projects, coupled with his high ambitions, quickly positioned him as the face of several projects. After presenting at a national research conference, many of the adult attendees seemed to empathize with his gripes about learning how to use a statistical package to analyze his research data. Upon arriving home Mikal asked us to mentor him and help deepen his data analysis of schools to better position himself as a professional on research related to youth. An avid follower of Internet sensation media mogul and speaker Gary Vee (Gennady Alexandrovich Vaynerchuk), Mikal studied his videos, which inspired him to upgrade his public speaking skills. Soon he was the opener at our public speaking events. Mikal, with his passion and charm, not only immediately engages his audience but also paves the way for his peers by warming up the crowd—imbuing them with more confidence to speak.
                </p>
                <p style="font-size: 1rem; line-height: 1.6; color: #c9c9c9; text-align: left; margin-top: 1rem;">
                    Clearly, we needed to support Mikal’s ambition with resources to help him advance, including access to technology and software to engage in research activities, professional photos for the speaking engagements, and award profiles, as well as personal introductions to world-renowned education scholars from schools including the University of Southern California; Teachers College, Columbia University; The Ohio State University; and the University of Michigan.
                </p>
                <p style="font-size: 1rem; line-height: 1.6; color: #c9c9c9; text-align: left; margin-top: 1rem;">
                    His desire to always go above and beyond what his project activities call for embodies what we want to do for all our youth: support them in developing into leaders who achieve scholastically in ways that also elevate their culture and community. Mikal attracted more students and more attention in general to our youth-led activities. And in turn we supported his professional presence, leveraged our network, and endorsed him for jobs and college admissions.
                </p>
                <p style="font-size: 1rem; line-height: 1.6; color: #c9c9c9; text-align: left; margin-top: 1rem;">
                    It came as no surprise to us that Mikal was recruited for a full ride by nationally recognized and highly selective research universities.
                </p>
            </div>

        </div>
    </div>`;